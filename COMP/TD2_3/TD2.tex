\documentclass[a4paper, 11pt]{article}
\usepackage[T1]{fontenc}
\usepackage[utf8]{inputenc}
\usepackage[french]{babel}

\usepackage{authblk}
\usepackage{hyperref}
\usepackage{graphicx}
\usepackage{listings}
%\usepackage{amsmath}
\usepackage[cmex10]{amsmath}
\usepackage{amssymb}
\usepackage{fullpage}
\usepackage{float}
\usepackage{url} \urlstyle{sf}

\usepackage{dashrule}
\usepackage{tikz}
\usetikzlibrary{arrows,automata}

\usepackage{color}
\usepackage{subcaption}
\newcommand{\alert}[1]{\color{red}{#1}}
\newcommand{\greyout}[1]{\color{gray}{#1}}

\title{TD2 et 3 COMP}
\author{}

\begin{document}
	\maketitle
	
\section{Exercice 1}
	
\subsection{Question 1}
	
	On ajoute un attribut synthétisé entier \textit{val} qui représente la valeur 
	de l'expression. L'attribut \textit{val} de la racine de l'arbre correspond à 
	la valeur de l'expression.
	
	On définit ainsi le calcul de \textit{val} :
	\begin{center}
		\begin{tabular}{l | r}
			$\mathit{chiffre} \rightarrow c $&$ \mathit{chiffre.val} = c $\\
			$\mathit{exp} \rightarrow \mathit{chiffre} $&$ \mathit{exp.val} = 
			\mathit{chiffre.val}$ \\
			$\mathit{exp}_1 \rightarrow (+ \mathit{exp}_2 \mathit{exp}_3) $&$ 
			\mathit{exp}_1\mathit{.val} =	\mathit{exp}_2\mathit{.val} + 
			\mathit{exp}_3\mathit{.val}$ \\
			$\mathit{exp}_1 \rightarrow (* \mathit{exp}_2 \mathit{exp}_3) $&$ 
			\mathit{exp}_1\mathit{.val} =	\mathit{exp}_2\mathit{.val} \times 
			\mathit{exp}_3\mathit{.val}$
		\end{tabular}
	\end{center}
	
	\subsection{Question 2}
	
	\tikzstyle{lien}=[->,>=stealth,rounded corners=2pt,thick]
	\tikzset{individu/.style={draw,thick,fill=#1!25},
		individu/.default={white}}
	
	\begin{figure}[h!]
		\centering
		\begin{tikzpicture}
		\node [lien,individu=white] {($+$ $\mathit{exp}$ $\mathit{sexp}$)} [sibling 
		distance=4cm,level distance=1.5cm]
		child { 
			node [lien,individu=blue]{$1$} [sibling distance=3cm]
		}
		child { 
			node [lien,individu=white]{$\mathit{exp}$ $\mathit{sexp}$} [sibling 
			distance=4cm]
			child { 
				node [lien,individu=white]{($*$ $\mathit{exp}$ $\mathit{sexp}$)} [sibling 
				distance=2cm]
				child { 
					node [lien,individu=blue]{$2$} 
				}
				child { 
					node [lien,individu=white]{$\mathit{exp}$} 
					child { 
						node [lien,individu=blue]{$3$} 
					}
				}
			}
			child { 
				node [lien,individu=white]{$\mathit{exp}$ $\mathit{sexp}$} [sibling 
				distance=2cm]
				child { 
					node [lien,individu=blue]{$4$} 
				}
				child { 
					node [lien,individu=white]{$\mathit{exp}$} 
					child { 
						node [lien,individu=blue]{$3$} 
					}
				}
			}
		};
		\end{tikzpicture}
		\caption{Arbre de dérivation de l'expression $(+1(*23)43)$}
		\label{fig:ast}
	\end{figure}
	
\subsection{Question 3}

Pour calculer l'attribut \textit{val} d'une règle \textit{sexp}, on a besoin de 
connaître l'opérateur concerné.

Pour ce faire, on ajoute un attribut hérité \textit{op} qui correspond à un 
opérateur : $+$ ou $*$.

On étend le calcul de \textit{val} :
\begin{center}
	\begin{tabular}{l | r}
		$\mathit{sexp} \rightarrow \mathit{exp}$ & $\mathit{sexp.val} = 
		\mathit{exp.val}$\\
		\end{tabular}
\end{center}

Et on définit ainsi le calcul de \textit{op} :
\begin{center}
	\begin{tabular}{l | r}
		$\mathit{exp}_1 \rightarrow (+ \mathit{exp}_2\quad \mathit{sexp}) $&$ 
		\mathit{sexp.op}= + $\\
		$\mathit{exp}_1 \rightarrow (*\mathit{exp}_2\quad \mathit{sexp}) $&$ 
		\mathit{sexp.op}= * $\\
		$\mathit{sexp}_1 \rightarrow \mathit{exp}\quad\mathit{sexp}_2$ & 
		$\mathit{sexp_1.val} = \mathit{sexp_1.op}(\mathit{exp.val}, 
		\mathit{sexp_2.val})$\\
	\end{tabular}
\end{center}

Alors dans un premier temps, on propage les opérateurs (en descendant dans 
l'arbre syntaxique). Puis, en remontant dans l'arbre syntaxique, on calcule la 
valeur de l'expression.

\subsection{Question 3 en une seule vague}
On peut n'utiliser que des attributs synthétisés. On définirait alors un
attribut synthétisé $\mathit{val\_list}$, qui contiendrait la liste des valeurs
des arguments d'une règle $\mathit{sexp}$. Alors, au moment d'évaluer une
expression $\mathit{exp}_1 \rightarrow (op\quad \mathit{exp}_2\quad
\mathit{sexp}) $, on implémente le calcul sous la forme d'une fonction
$\mathit{fold}$ qui appliquerait l'opérateur sur tous les éléments de la liste.


\section{Exercice 3}
    
\subsection{Question 1}
\begin{figure}[h!]
    \centering
    \begin{tikzpicture}[->,>=stealth',rounded corners=2pt,shorten >=1pt,auto,node distance=2.5cm,
                        thick,node/.style={draw,font=\sffamily\large\bfseries}]

    \node[node] (1)                      {\textit{DecTypes}};
    \node[node] (2)  [below of=1]        {\textit{Declarations}};
    \node[node] (3)  [below of=2]        {\textit{DefType} ; \textit{Declarations'}};
    \node[node] (4)  [below of=3]        {\underline{struct} ident \{\textit{ListeChamps}\}};
    \node[node] (5)  [below of=4]        {\textit{Champ} \textit{ListeChamps'}};
    \node[node] (6)  [below of=5]        {\textit{Type}};
    \node[node] (71) [below left of=6]   {\underline{int}/\underline{float}/\underline{char}};
    \node[node] (72) [below right of=6]  {\underline{struct} ident};

    \path[every node/.style={font=\sffamily\small}]
        (1)  edge              node [left]                  {\textit{Declarations}.decl = $\emptyset$} (2)
        (2)  edge              node [align=right] [left]    {\textit{DefTypes}.decl = \textit{Declarations}.decl} (3)
        (2)  edge              node [align=left] [right]    {\textit{Declarations'}.decl = if DefType.correct \\then Declarations.decl $\cup$ DefType.ident \\else Declarations.decl} (3)
        (3)  edge [bend right] node [left]                  {\textit{ListeChamps}.decl = \textit{DefTypes}.decl} (4)
        (4)  edge [bend right] node [align=right] [left]    {\textit{Champ}.decl = \textit{ListChamps}.decl\\ \textit{ListChamps'}.decl = \textit{ListChamps}.decl} (5)
        (5)  edge [bend right] node [left]                  {\textit{Type}.decl = \textit{Champ}.decl} (6)
        
        (71)  edge [bend left]  node [left]                 {\textit{Type}.correct = True} (6)
        (72)  edge [bend right] node [right]                {\textit{Type}.correct = (ident $\in$ \textit{Type}.decl)} (6)
        (6)  edge [bend right]  node [right]                {\textit{Champ}.correct = \textit{Type}.correct} (5)
        
        (5)  edge [bend right]  node [right]                {\textit{ListeChamps}.correct = \textit{Champ}.correct $\land$ \textit{ListeChamps'}.correct} (4)
        (4)  edge [bend right]  node [align=left] [right]   {\textit{DefType}.correct = \textit{ListeChamps}.correct\\ \textit{DefType}.ident = ident} (3)
        ;
    \end{tikzpicture}
\end{figure}
    
\subsection{Question 2}

    
\begin{figure}[h!]
    \centering
    \begin{tikzpicture}[->,>=stealth',rounded corners=2pt,shorten >=1pt,auto,node distance=2.5cm,
                        thick,node/.style={draw,font=\sffamily\large\bfseries}]

    \node[node] (1)                      {\textit{DecTypes}};
    \node[node] (2)  [below of=1]        {\textit{Declarations}};
    \node[node] (3)  [below of=2]        {\textit{DefType} \textit{Declarations'}};
    \node[node] (4)  [below of=3]        {\underline{struct} ident \{\textit{ListeChamps}\}};
    \node[node] (5)  [below of=4]        {\textit{Champ} ; \textit{ListeChamps'}};
    \node[node] (6)  [below of=5]        {\textit{Type}};
    \node[node] (71) [below left of=6]   {\underline{int}/\underline{float}/\underline{char}};
    \node[node] (72) [below right of=6]  {\underline{struct} ident};

    \path[every node/.style={font=\sffamily\small}]
        (1)  edge              node [left]                  {\textit{Declarations}.decl = $\emptyset$} (2)
        (2)  edge              node [align=right] [left]    {\textit{DefTypes}.decl = \textit{Declarations}.decl} (3)
        (2)  edge              node [align=left] [right]    {\textit{Declarations'}.decl = if DefType.correct \\then Declarations.decl $\cup$ DefType.ident \\else Declarations.decl} (3)
        (3)  edge [bend right] node [left]                  {\textit{ListeChamps}.decl = \textit{DefTypes}.decl} (4)
        (4)  edge [bend right] node [align=right] [left]    {\textit{Champ}.decl = \textit{ListChamps}.decl\\ \textit{ListChamps'}.decl = \textit{ListChamps}.decl} (5)
        (5)  edge [bend right] node [left]                  {\textit{Type}.decl = \textit{Champ}.decl} (6)
        
        (71)  edge [bend left]  node [left]                 {\textit{Type}.correct = True} (6)
        (72)  edge [bend right] node [right]                {\textit{Type}.correct = (ident $\in$ \textit{Type}.decl)} (6)
        (6)  edge [bend right]  node [right]                {\textit{Champ}.correct = \textit{Type}.correct} (5)
        
        (5)  edge [bend right]  node [right]                {\textit{ListeChamps}.correct = \textit{Champ}.correct $\land$ \textit{ListeChamps'}.correct} (4)
        (4)  edge [bend right]  node [align=left] [right]   {\textit{DefType}.correct = \textit{ListeChamps}.correct\\ \textit{DefType}.ident = ident} (3)
        ;
    \end{tikzpicture}
\end{figure}

\end{document}
